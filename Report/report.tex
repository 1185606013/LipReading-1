\documentclass{article}[10pt]
\usepackage[utf8]{inputenc}
\usepackage[english]{babel}
\usepackage[margin=0.8in]{geometry}

\usepackage{multicol}

\author{
  Héctor Otero Mediero\\      \texttt{hoterome@uci.edu}
  \and
  Nikita Samarin\\      \texttt{nsamarin@uci.edu}
}
\title{Lip Reading}

\begin{document}
\maketitle

\begin{multicols}{2}

\section{Introduction}
Why is Lip Reading important?
\section{Description of the Problem}
Describe in detail going from video to what output.
\section{Previous Work}
Talk about papers in the table.
\section{Dataset}
Description of OuluVS and excuses of why don't we have a larger dataset.
Videos, frames, information about box coordinates, and sound.
How many frames in total, dimensions of the images.
Extra images from other datasets.
\section{Hardware}
Macbook Pro: 2.4GHz dual-core Intel Core i5 processor, 8GB DDR3L, 256GB
\section{Technical Approach}
\subsection{Using video frames}

Our first attemps at tackling the problem followed a pure lip reading approach,
this is, using only a sequence of still images (video frames) from a subject to
predict the sentence that's said in the video. Previous works in the field used
3D Convolutional Neural Networks to process the sequence of images with
succesful results, but our hardware didn't fit well this tasks as the
computation requirements is large and a single epoch for processing 500 videos
took around 2 hours. Due to the impossibility of following this path, we
flattened the input to be able to process each image as a 1D vector and be able
to process it with a Recurrent Neural Network or a 2D convolutional Network
(after forming a matrix with all the different frames). The results were not
good as the dimensions of the individual images was too big in comparison with
the amount of examples yielding networks with too many parameters that
overfitted the data.

\paragraph{Lip Segmentation} ~\\

In order to reduce the dimensionality of the data, we decided to limit the
section of the photo used as input. As it's obvious, the totality of the
information stored in an image comes from the mouth of the subject and since the
space it occupies is rather small in comparison with the image size it does a
great job at condensing the information and reducing the amount of parameters in
the network.\\

Since solving image segmentation problems using Neural Networks has been
thoroughly studied, we designed our own. The problem at hand was finding the
coordinates of the top left corner and the width and height of a bounding box
that surrounded the subject's mouth. The dataset we were working with included
labeled data of each one of the frames extracted from the videos with this
information. \\

As it can be seen in Figure 1 the architecture used is composed by alternating
2D Convolutional layers with MaxPooling2D layers and a final Dense layer that
completes the regression task by predicting 4 values from each image. The
intuition behind this architecture is that, as in Visual Computing techniques,
the convolutional layers will be used to extract features at different levels of
detail and the pooling layers help at reducing further the dimensionality of the
image. Finally, the dense layer with a linear activation produces an unbounded
value, ideal for the regression task.\\

%TODO Include Figure 1

The results obtained are really accurate. After 25 training epochs and using 80\%
of the data for training and the rest for test, we obtain a Mean Squared Error of
95. The semantics of this error for our case are that, on average, the predictions
for the coordinates of the box and the labeled data differ in only 9 pixels, which
for a 576x720 images represents less than 2\% of the width and height.\\

%TODO Include image bounding boxes

Some of the reasons behind this accurate prediction are the availability of a
large set of images to work with and the fact that the mouths are centered along
the X-axis.

\paragraph{RNN Architecture} ~\\

Using the previous information to crop the images, we tried feeding them to a
recurrent neural network due to the fact that there's a relation among the
images that can be expressed as a sequence.  Since we're in front of a
classification task, we use a Dense layer at the top with a softmax activation
function that returns values for the different classes that can be interpreted
as the probability of an example of belonging to a certain class.\\

Both types of RNNs available in Keras, LSTMs and SimpleRNNs, were used to test
whether there was a need for establishing a relation between values distant in
the video sequence. The architecture that yielded the best results used the
latter and is shown below. The layer configuration responds to the scarcity of
the data (just 1000 videos), which forced us to constrain our network to a small
amount of parameters, and the subsequent need to avoid overfitting, which led us
to include dropout layers after each of the recurrent ones and try different
types of regularization (L2 with a value of 0.01 providing the best
performance).\\

%TODO Insert image of RNN

The results obtained barely improved a random classification of the data with a
16\% accuracy using categorical crossentropy as loss function for the model. We
think that a solution could be found using this method due to the semantic
relation between the images as explained previously but recurrent networks that
can process this kind of data need a lot of parameters and, in our case, the
relation with the amount of input data was far from ideal. 2D Convolutional
networks were also tested but they produced worse results.

\subsection{Using video sound}

Due to the poor results obtained using just images, we decided to try solving
the same problem (going from a video to the sentence said) but using the sound
instead since in most cases where lip reading can be used, sound is available
too (even if it's present with noise). Intuitively, the same information is
stored both in the video and in the sound, the second one being more dense and
simpler than the first.

\paragraph{Data Preprocessing \& Augmentation} ~\\

In order to reduce the problems found with the previous approach, we opted to
increase the amount of training examples. The audio present in the videos was
noisy so we tried different approaches to lessen the effects that they could
provoke in our network, generating new audios by applying a rolling-mean filter
on the audio with a window size of 10 making reducing the noise effectively.\\

Apart from the noise, we tested different representations of the data, integer
and floating point, and a different number of channels, estereo or mono, choosing
in both cases the second option as it produced better results.

\paragraph{CNN Architecture} ~\\

Facing again a vectorial representation of our input data (in this case
amplitude values for the audio) we could again choose between a recurrent or a
convolutional approach. By representing the data we saw that audios only
differed one from the other on the amount of words pronounced and that in terms
of amplitude they were hardly separable. Because of this we chose convolutional
layers as they treat the complete vector at once and could obtain they
aforementioned features better than a recurrent network. \\

Using a layer structure (Figure X) similar to the one built to predict the
bounding box but this time using 1D Convolution we obtained our best results at
predicting the phrase said in a video, a 35\% accuracy. The activation function
for the hidden layers was RELU as it showed a faster convergence than the rest
and the Adam (Adaptive Moment Estimation) optimizer homogenized the reduction of
the loss in comparison to other optimizers that didn't stop the loss from
increasing and decreasing greatly in the validation set.


\subsection{Conclusions \& Future Work}
\end{multicols}
\end{document}
